
%%% use two columns and PHYSCON 2011 format
\documentclass[twocolumn]{CAP-journal}


\usepackage{graphicx,array}
\usepackage{amssymb,amsmath}

% If you want to use BibTex, then uncomment the following command:
%\usepackage[dcucite]{harvard}


\title{INSTRUCTIONS FOR THE PREPARATION AND SUBMISSION OF PAPERS \\
FOR Cybernetics And Physics USING \LaTeX2e\ }

%% Use the \author command to put authors horizontally on one row.
%% Separate the authors with the \and command. Put as may authors as possible on one row.
%% If necessary, then use a second \author command as shown below, which will put additional authors on another row.

%%% first two or three authors
\author{ \twlsfb Firstname1 Familyname1
    \affiliation{
      Department1\\
      University1\\
      Country1\\
      name1@university1.country1
    }
    \and \twlsfb Firstname2 Familyname2
    \affiliation{
      Department2\\
      University2\\
      Country2\\
      name2@university2.country2
    }
}

%%% another two or three authors, uncomment if necessary
%\author{ \twlsfb C. Third
%    \affiliation{
%      Institute\\
%      University\\
%      Country\\
%      a.name@university.country
%    }
%    \and \twlsfb D. Fourth
%    \affiliation{
%      Institute \\
%      University\\
%      Country\\
%      anothername@university.country
%    }
%    \and \twlsfb E. Fifth
%    \affiliation{
%      Institute \\
%      University\\
%      Country\\
%      fifth@university.country
%    }
%}



\begin{document}

\maketitle


\begin{abstract}
This article illustrates the preparation of a paper for Cybernetics And Physics (CAP) journal
using \LaTeX2e. The \LaTeX\ class \verb+CAP-journal.cls+ and
template \verb+CAPauthor.tex+ that create this article are
available on http://cap.physcon.ru. You should
only modify the \verb+CAPauthor.tex+ file and not modify \LaTeX\
class \verb+CAP-journal.cls+ while preparing your CAP paper.
%These files come with no guarantee and might contain bugs. Still, they
%may provide the user some help in preparing a paper for the PHYSCON~2011 Conference.
\end{abstract}


\begin{keywords}
 No more than six key words must be given.
\end{keywords}


\section{Introduction}
Your paper for the CAP journal  must be prepared in
\textbf{English}. To ensure that your paper will be reproduced
clearly and in the proper size and form, please observe the
following instructions. In addition to length and format
restrictions given below, all papers will be reviewed
% at least by two independent reviewers
and must meet technical standards.


\section{Manuscripts}
It is advised to put some text between a section heading and a subsection heading.

\subsection{Organization}
Your paper should begin with an abstract, followed by key words, and then an introduction. At the end, it should
feature a conclusion and a list of references. Acknowledgements (if any) may precede the references.

\subsection{Paper length and types}
The length of the title of your paper should not exceed two lines.
The maximum length of your abstract is to be 200 words. Three paper types are admitted: survey,
regular and brief paper.  The length of a regular paper,
including figures and tables, should not exceed \textbf{twelve (12) two-column pages.}
 The length of a brief paper
%including figures and tables,
 should be \textbf{4-6 two-column pages.} The length of survey paper has no restriction.
 However, the authors are advised to contact the Editorial Board before writing a survey paper.
%(for final paper this number may be changed).
The length of the paper should be modified if more than about two thirds of the last page remains empty.

\subsection{Layout}
The layout of the paper is taken care for by the
 \verb+CAP-journal.cls+ class file and template \verb+CAPauthor.tex+.
%Please check whether the PDF file gives the desired result.

\subsection{Symbols}
Symbols and acronyms should be defined the first time they appear. Use the International System (SI) of units.

\subsection{Page Numbering}
Do not number pages. 
This should be taken care for by the \verb+CAP-journal.cls+ class file.


\section{Illustrations}
For the online version of the journal both color and black-and-white drawings, tables and photographs are acceptable. Please
check whether they are still legible when printed in black-and-white (for printed version of the journal). Your illustrations must be placed within the \verb+figure+
environment, accompanied by captions and inserted in the text near the place where they are mentioned. Use the same or
similar font sizes in your illustrations as in the manuscript. Check whether the smallest lettering, such as
subscripts, is still legible. Captions should be made with the \verb+\caption+ command and should explain illustrations
without a further reading of the paper text.
\begin{figure}[t]
\begin{center}
\setlength{\unitlength}{0.012500in}%
\begin{picture}(115,35)(255,545)
\thicklines \put(255,545){\framebox(115,35){}} \put(275,560){Beautiful Figure}
\end{picture}
\end{center}
\caption{An example figure.} \label{figurelabel}
\end{figure}


\section{Equations}
All equations should be centered. The equations that are referred to should be numbered. Use the \verb+equation+ environment, like the example given below.
\begin{equation}
 \left.\begin{array}{l}
  \nabla \times E = -\frac{\partial B}{\partial t}\\
  \nabla \times H = J + \frac{\partial D}{\partial t}\\
 \end{array}\right\}
\end{equation}


\section{Bibliography}
List all bibliographical references at the end of the paper in the
manner shown (listing in alphabetic order with respect to last
name of first author). Refer to them in the text as follows:
\cite{Wang} and~\cite{Paul,Sato}. If you want to use BibTeX, then
follow the instructions in \verb+CAPauthor.tex+.

\section{Compilation}
It might occur that the paper size is incorrect when compiling the
dvi-file to a pdf-file (the banner falls partly off the page). If
this problem occurs, then compile the \verb+CAPauthos.dvi+ file
with DVIPS using the following options

\verb+dvips -t a4 -P pdf physcon+ and subsequently compile to PDF.
Alternatively, you can add the options \verb+-sPAPERSIZE=a4+ to
the ps2pdf command.



\section{Paper Submission}
Your paper should uploaded though Submission site
http://cap.physcon.ru \textbf{in PDF format}
 %by \textbf{January 31, 2011}.
  Faxed and emailed submissions will not be accepted. To
maintain the same look when viewing or printing your paper on
another machine, you should embed all fonts used in the PDF file.
When creating a PDF file, click $<$Properties$>$, $<$Adobe PDF
Setting$>$, $<$Job Option Edit$>$ and then $<$Embed All
Fonts$>$.\textbf{The size of the PDF file is not allowed to exceed
2.0 MB.} Upon acceptance of the paper you may be requested to submit 
the source files too.


\section{Conclusion}
A carefully prepared manuscript that is acceptable without modification will make a better presentation of your work
and spare us unnecessary delay in editing the papers.
\section*{Acknowledgements}
This template was updated from the template prepared for ENOC
2005, Eindhoven by Remco I. Leine, Center of Mechanics IMES, ETH.
Modified  for CAP journal based on experience of Physcon2005-Physcon2011 conferences.
\begin{thebibliography}{}

\bibitem[Paul, 1992]{Paul} Paul, C.~R. (1992). {\em Introduction to Electromagnetic Compatibility}. Wiley-Intersciences. New~York.

\bibitem[Sato, 1989]{Sato} Sato, R. (1989). EMC - The past, present and future. In {\em Proc. Int. Symp. on Electromagn. Compat.} Nagoya, Japan, Sept. 8-10. pp.~1--9.

\bibitem[Wang, Sasabe and Fujiwara, 2002]{Wang} Wang, J., Sasabe, K. and Fujiwara, O. (2002)  A simple method for predicting common-mode radiation from a cable attached to a conducting enclosure. {\em \mbox{IEICE} Trans. Commun.}, \textbf{E85-B}(7), pp.~1360--1367.

\end{thebibliography}
%% If you want to use Bibtex, then remove the thebibliography environment above,
%% and uncomment the \bibliography command below as well as \usepackage[dcucite]{harvard} at the beginning of the document.
%\bibliography{physconref}
\bibliographystyle{physcon}


\appendix       %%% starting appendix, remove if no appendix exists
\section*{Appendix A\; Head of an Appendix}
Avoid using appendices.


\end{document}
